\documentclass[12pt]{article}

%% preamble: Keep it clean; only include those you need
\usepackage{amsmath}
\usepackage[margin = 1in]{geometry}
\usepackage{graphicx}
\usepackage{booktabs}
\usepackage{natbib}

% for space filling
\usepackage{lipsum}
% highlighting hyper links
\usepackage[colorlinks=true, citecolor=blue]{hyperref}


%% meta data

\title{Proposal: How to Get a Job in Data Science Using Data Science}
\author{Melanie Desroches\\
  Department of Statistics\\
  University of Connecticut
}

\begin{document}
\maketitle


\paragraph{Introduction}


As a senior in college, it is only natural that I have been considering what my career will look like post-graduation.
As a result, I have been looking at job applications to see what employers are looking for in candidates and if I qualify. 
It is very easy to get overwhelmed by all the skills and qualifications that you need to have in order to get the job that 
you want. In this project, I’ll explore the different types of data science roles by clustering job postings based on skill 
requirements and job types. Using a dataset that consists of data science related job postings from Glassdoor, the goal is to
identify natural groupings, such as data engineering, data analysis, and machine learning roles, based on the skills each type 
of job requires.

\paragraph{Specific Aims}
Formulate your research question;
translate your research question into statistical/data science questions

The goal of this project is to identify what factors employers are looking for in a candidate for a data science job.
There are different types of jobs in the realm of data science, such as data engineering, data analysis, machine learning 
engineering, etc. Based on the different types of roles, what skills are the most prominent. For example, do data engineering
roles tend to require AWS knowledge? Python? Tableau? Based on the clusters, the aim is to uncover each “profile” and discuss 
which skill combinations are common. These profiles will illustrate the typical skill sets needed across different data science job types 
and levels.

\paragraph{Data}

The data set I will be using is based on job postings on Glassdoor. The data can be found on Kaggle. The dataset already has
some initial cleaning done. The columns that will be of the most interest to me are Job Description (which is the full description
on the job posting), python, excel, hadoop, spark, aws, tableau, big data (which are all boolean variables used to determine
if it is a relevant skill to the job), and information about the company that is hiring (name, industry, etc).


\paragraph{Research Design and Methods}

To perform my analysis, I will be apply a clustering algorithm such as K-means or Hierarchical Clustering. The first
step will be processing and cleaning the dataset. This will involve looking for any missing values, removing any jobs 
that seem irrelevant (meaning they do not involve data science), and dropping unnecessary columns. This step will also involve some 
feature engineering. Using the job descriptions provided, skills can be pulled and turned into new columns. Once the data is 
prepared, the appropriate clustering algorithm can be applied. Based on the results, data visualizations can then be generated.

\paragraph{Discussion}
What are the most challenge parts of the task?
What are the limitations of your work? What is your fall-back plan if
something unexpected happens?


\bibliography{../manuscript/ref}
\bibliographystyle{chicago}

\end{document}
