\documentclass[12pt]{article}

%% preamble: Keep it clean; only include those you need
\usepackage{amsmath}
\usepackage[margin = 1in]{geometry}
\usepackage{graphicx}
\usepackage{booktabs}
\usepackage{natbib}

% for space filling
\usepackage{lipsum}
% highlighting hyper links
\usepackage[colorlinks=true, citecolor=blue]{hyperref}


%% meta data

\title{Proposal: Something Interesting}
\author{Melanie Desroches\\
  Department of Statistics\\
  University of Connecticut
}

\begin{document}
\maketitle


\paragraph{Introduction}
Background about your research; touch the three questions to be addressed in an
introduction; cite relevant refernces~\citep[e.g.,][]{dwivedi2017analysis}.

As a senior in college, it is only natural that I have been considering what my career will look like post-graduation.
As a result, I have been looking at job applications to see what employers are looking for in candidates and if I qualify. 
It is very easy to get overwhelmed by all the skills and qualifications that you need to have in order to get the job that 
you want. In this project, I’ll explore the different types of data science roles by clustering job postings based on skill 
requirements and job types.

\paragraph{Specific Aims}
Formulate your research question;
translate your research question into statistical/data science questions

The goal of this project is to identify what factors employers are looking for in a candidate for a data science job.

\paragraph{Data}
The data set I will be using is based on job postings on Glassdoor. The data can be found on Kaggle. 

\lipsum[3]

\paragraph{Research Design and Methods}
What design or methods will you use?
Cite relevant references~\citep[e.g.,][]{wild2004global}.

To perform my analysis, I will be apply a clustering algorithm such as K-means or Hierarchical Clustering. 

\paragraph{Discussion}
What are the most challenge parts of the task?
What are the limitations of your work? What is your fall-back plan if
something unexpected happens?

\lipsum[5]

\bibliography{../manuscript/ref}
\bibliographystyle{chicago}

\end{document}
